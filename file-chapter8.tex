\chapter{Objetivos secundarios del cliente}

Como ya se mencion\'{o} en la secci\'{o}n \ref{obj-cliente} aparte del objetivo principal del proyecto tambi\'{e}n se citaron ciertos objetivos <<extra>> o necesidades del cliente una vez realizada la segmentaci\'{o}n. Este cap\'{i}tulo se centra en recoger estas necesidades y explicar en general c\'{o}mo se han llevado a cabo. Adem\'{a}s, con estas tareas tambi\'{e}n se podr\'{a} ver el verdadero prop\'{o}sito de la segmentaci\'{o}n realizada; el poder encontrar los objetos, analizarlos y poder extraer conclusiones de \'{e}stos en el \'{a}mbito en el que se est\'{e} trabajando. Las tareas a realizar se deber\'{a}n de hacer de manera autom\'{a}tica y para cualquier imagen.

Como recordatorio, en la aproximaci\'{o}n al algoritmo \textit{level set} la funci\'{o}n $\phi$, que representa el \textit{level set}, est\'{a} definida en una cuadr\'{i}cula (del mismo tama\~{n}o que la imagen a segmentar) y puede tomar los valores \{-3, -1, 1, 3\} (v\'{e}ase la ecuaci\'{o}n \ref{phi-levelSetEcuation}). Una vez que se haya realizado la segmentaci\'{o}n todos los valores negativos representar\'{a}n las islas, mientras que los valores positivos representar\'{a}n el fondo. Teniendo esta representaci\'{o}n en cuenta se realizar\'{a}n las tareas que se desarrollan a lo largo de este cap\'{i}tulo.

El orden de presentaci\'{o}n de cada tarea ser\'{a} el presentado en la secci\'{o}n \ref{obj-cliente}.
\section{Recubrimiento}

Teniendo en cuenta lo presentado en la introducci\'{o}n de este cap\'{i}tulo sobre la funci\'{o}n $\phi$ esta tarea es realmente sencilla. Dado que las islas tendr\'{a}n un valor negativo y el fondo un valor positivo, \'{u}nicamente habr\'{a} que contar los valores negativos que haya en la funci\'{o}n $\phi$, es decir, en la matriz que representa esa funci\'{o}n. Se ha realizado una implementaci\'{o}n paralela de esta tarea mediante una operaci\'{o}n de reducci\'{o}n\protect\footnotemark como se presenta en el c\'{o}digo mostrado a continuaci\'{o}n. 
\footnotetext{Cada thread realiza una operaci\'{o}n con una variable local y al final se operan estos resultados locales para conseguir el resultado final}
\begin{lstlisting}
double ActiveContour::calculateCovering(int innerValue) const
{
	int sum=0,size=img_width*img_height,inValue;
	
	#pragma omp parallel for reduction(+:sum)
	for(int i= 0; i < size; i++)
	{
		if(innerValue < 0)
			if(phi[i] < 0 )	sum++;
		else
			if(phi[i] > 0 )	sum++;
	}
	double percentage = (sum * 100) / size;
	return percentage;
}	
\end{lstlisting}
\section{Extracci\'{o}n y conteo de islas}

Para realizar estas tareas primeramente se debe encontrar una isla de la imagen. Para ello, se busca por filas en la matriz $\phi$ el primer punto que pertenezca a una isla, es decir, que tenga un valor -1, que corresponder\'{i}a al borde de \'{e}sta. Una vez encontrada la isla se procede a su extracci\'{o}n. Para ello se recorre recursivamente los puntos vecinos de ese primer punto con un valor $\phi = -1$. Esta situaci\'{o}n siempre se dar\'{a} ya que el valor -1 corresponde a los puntos que pertenec\'{i}an a la lista $L_{in}$, la cual llegaba a delimitar el borde de la isla por completo. Mientras se realiza ese recorrido se cogen los m\'{a}ximos y m\'{i}nimos valores de las coordenadas de cada punto del recorrido. De esta manera, cuando se haya terminado por completo el recorrido a la isla, se encuadra y se pinta el borde.

Una vez obtenido el contorno de la isla y se haya encuadrado, se debe de <<pintar>> o rellenar por completo \'{e}sta. Esta tarea no es trivial, ya que la isla puede tener muchas formas distintas, y la idea es que esta operaci\'{o}n se realice para cualquier tipo de islas y de forma autom\'{a}tica. Estos algoritmos se denominan \textit{Flood fill algorithms}. Se ha escogido realizar una implementaci\'{o}n recursiva, de manera que se <<pinten>> todos los p\'{i}xeles vecinos de un p\'{i}xel interno dado de manera recursiva. De esta manera se escoge un vecino cercano al primer punto de la isla que pertenezca al interior de la isla, es decir, que tenga un valor $\phi = -3$ y se realiza el <<pintado>>.

Realizado lo anterior, se vuelve a buscar desde el punto de la primera isla otra nueva, de manera que se extraigan todas las islas de la imagen. La tarea de contar se hace mientras se van extrayendo las islas una a una. De esta manera el conteo se puede hacer simult\'{a}neamente cumpliendo el objetivo de extracci\'{o}n de las islas.

A forma de ejemplo se muestra en la figura \ref{extraccionIslaBuena} el proceso de extracci\'{o}n de las primeras tres islas de la figura \ref{buena1} utilizada en la experimentaci\'{o}n. Con prop\'{o}sito de reducir el ejemplo s\'{o}lo se ha puesto un trozo de esta \'{u}ltima imagen y se ha se\~{n}alado la posici\'{o}n de las tres primeras islas.


\begin{figure}[H]
	\captionsetup{justification=centering}
	\centering
	\includegraphics[width=1\textwidth]{./imagenes/extraccionIslaBuena}
	\caption{Proceso de extracci\'{o}n de las islas de la figura \ref{buena1} }	
	\label{extraccionIslaBuena}
\end{figure}



\section{Densidad}

Al cliente le interesa saber la densidad de las islas interiores y exteriores por separado, y realizar la media de los dos valores. Esta tarea se podr\'{a} hacer una vez realizada la extracci\'{o}n de las islas, es decir, la anterior tarea. El c\'{a}lculo de la densidad de todas las islas de la imagen ser\'{i}a sencilla, ya que una vez que se hayan extra\'{i}do \'{u}nicamente habr\'{a} que realizar la divisi\'{o}n: n\'{u}mero de islas entre el \'{a}rea total de la imagen.

 En cuanto a la primera cuesti\'{o}n, se deber\'{i}an de identificar las islas que est\'{a}n <<cortadas>> por los bordes ya que estar\'{a}n incompletas. La soluci\'{o}n es realizar un conteo de las islas que no est\'{a}n en el borde a la par que se extraen, observando si alguna coordenada perteneciente al borde de las islas est\'{a} tocando el borde de la imagen. Una vez que tengamos el conteo se realizar\'{a} la misma divisi\'{o}n pero esta vez entre el n\'{u}mero de islas que no tocan el borde y el \'{a}rea del marco en la que est\'{a}n circunscritas las islas. Para finalizar se realiza la media entre las dos densidades obtenidas.

\section{Power Spectral Density (PSD)}

\textit{Power Spectral Density}, o la densidad espectral de potencia en castellano, es una medida de la intensidad de la potencia de una se\~{n}al en el dominio de la frecuencia, informando de la distribuci\'{o}n de sus distintas componentes frecuenciales en el espectro. Para el cliente de este proyecto, este c\'{a}lculo ayudar\'{a} a evaluar las caracter\'{i}sticas de las islas que contiene la imagen que, como ya se ha mencionado anteriormente, est\'{a}n producidas por ciertos experimentos. El resultado del PSD es una especie de <<mapa>> sobre las intensidades que tiene la imagen. El cliente tiene una herramienta en Java (desarrollada por Nestor Ferrando y adaptada por Joseba Alberdi) que realiza el PSD de una imagen representada en forma de <<tienda de campa\~{n}a>> (v\'{e}ase las figuras \ref{tienda} y \ref{tienda2}). Para crear esta representaci\'{o}n de <<tienda de campa\~{n}a>> se deber\'{a} generar un fichero <<.txt>> de aquello que se quiera calcular el PSD, por lo que se tendr\'{a}n que adaptar las im\'{a}genes extra\'{i}das de la segmentaci\'{o}n y generar ese formato de manera que se pueda realizar el c\'{a}lculo con dicha herramienta.

El cliente necesita sacar conclusiones sobre varios c\'{a}lculos de PSD y la comparaci\'{o}n entre \'{e}stos. Para ello, se han tenido que realizar varias tareas:

\begin{itemize}
	\item Se ha realizado una imagen, del mismo tama\~{n}o que la imagen original, de todas las islas encontradas en la segmentaci\'{o}n, incluyendo las islas de los bordes y posicionando todas ellas en la posici\'{o}n que se encontraban en la imagen original. Adem\'{a}s, se ha realizado otra imagen aparte de las mismas caracter\'{i}sticas que la anterior pero excluyendo las islas de los bordes. De la misma manera, se han generado sus correspondientes ficheros <<.txt>> para poder realizar el c\'{a}lculo del PSD con las dos im\'{a}genes.
	\item Se han centrado las im\'{a}genes de todas las islas extra\'{i}das, se han convertido en im\'{a}genes cuadradas y todas del mismo tama\~{n}o (tama\~{n}o de la isla m\'{a}s grande). Para cada una de estas islas se ha generado su fichero <<.txt>> correspondiente. Adem\'{a}s, se han identificado las islas que <<no>> est\'{a}n <<cortadas>> por el borde de la imagen y se han generado nuevamente sus correspondientes ficheros <<.txt>>.
	\item Se ha generado otra nueva imagen por cada isla, pero esta vez se ha pintado la isla en el lugar donde se encontraba en la imagen original, por lo tanto, tendr\'{a} el mismo tama\~{n}o que esta \'{u}ltima. Posteriormente se ha generado su fichero <<.txt>> correspondiente.
	\item Se ha generado otra nueva imagen por cada isla, aparte de las dos anteriores generadas, con el tama\~{n}o de la imagen original y centradas. De la misma manera tambi\'{e}n se han generado sus correspondientes ficheros <<.txt>>.  
\end{itemize}

Con los ficheros generados se generan las <<tiendas de campa\~{n}a>> que se pueden observar en las figuras \ref{tienda} y \ref{tienda2}. Despu\'{e}s, la herramienta del cliente crea el PSD a partir de estas representaciones. La figura \ref{tienda} corresponde a la <<tienda de campa\~{n}a>> utilizada para el c\'{a}lculo del PSD de la media de las islas de la figura \ref{result3}. 

\begin{figure}[H]
	\captionsetup{justification=centering}
	\centering
	\includegraphics[width=.6\textwidth]{./imagenes/psd}
	\caption{Ejemplo de <<tienda de campa\~{n}a>> de la media de las islas de la figura \ref{result3}}	
	\label{tienda}
	Fuente: Joseba Alberdi
\end{figure}

A modo de ejemplo, se muestra en la figura \ref{tienda2} la <<tienda de campa\~{n}a>> de la octava isla encontrada en la segmentaci\'{o}n de la figura \ref{buena3}. En la figura \ref{tienda_aux} se muestra como ayuda el resultado de la segmentaci\'{o}n de la figura \ref{buena3} detallando la ubicaci\'{o}n exacta de la octava isla.

\begin{figure}[H]
	\captionsetup{justification=centering}
	\centering
	\begin{subfigure}[t]{2.6in}
		\centering
		\includegraphics[width=1.3\textwidth]{./imagenes/tiendaAux}
		\subcaption{}\label{tienda_aux}
	\end{subfigure}	
	\begin{subfigure}[t]{2.6in}
		\centering
		\includegraphics[width=1.45\textwidth]{./pdfs/surface_raw}	
		\subcaption{}\label{tienda2}
	\end{subfigure}
	\caption{Ilustraci\'{o}n de la octava isla }	
	Fuente: Joseba Alberdi
\end{figure}

Con estas representaciones la herramienta del cliente realiza los <<mapas>> correspondientes al PSD como se puede observar en la figura \ref{PSDmap}.

\begin{figure}[H]
	\captionsetup{justification=centering}
	\centering
	\includegraphics[width=0.5\textwidth]{./imagenes/PSDmap}
	\caption{Ejemplo de mapa de PSD}	
	Fuente: Joseba Alberdi
	\label{PSDmap}
\end{figure}

	
Como resultado de la extracci\'{o}n de toda la informaci\'{o}n de las islas y los PSD posteriormente calculados se ha generado un informe con toda la informaci\'{o}n obtenida. Este informe, creado por Joseba Alberdi y el creador de esta memoria, Daniel Franco, recoge la comparaci\'{o}n de diferentes c\'{a}lculos del PSD con el fin de que el cliente pueda observar las diferencias y sacar conclusiones de ello. El informe est\'{a} creado en ingl\'{e}s y se puede encontrar adjunto en el ap\'{e}ndice \ref{apendice2}. 