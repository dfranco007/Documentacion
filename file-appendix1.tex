\begin{appendices}

\chapter{Ciclos principales del algoritmo \textit{level set}}\label{apendice1}

\subsubsection{Primer ciclo}
\begin{lstlisting}

void ActiveContour::do_one_iteration_in_cycle1()
{
	// means of the Chan-Vese model for children classes ACwithoutEdges and ACwithoutEdgesYUV
	calculate_means(); // virtual function for region-based models
	
	hasOutwardEvolution = false;
	hasInwardEvolution = false;
	
	Lout.splitList(Splited_Lout,numThreads,sublistHead, sublistHeadPosition,0);
	Lin.splitList(Splited_Lin,numThreads,sublistHead, sublistHeadPosition,1);
	
	int tid;
	#pragma omp parallel private(tid)
	{
		tid = omp_get_thread_num();
		hasOutwardEvolutionInThread[tid]=false;     
		
		for( list<int>::iterator Lout_point = Splited_Lout[tid]->begin(); !Lout_point.end();)
		{
			if(compute_external_speed_Fd(*Lout_point, tid) > 0)
			{
				hasOutwardEvolutionInThread[tid]=true;
			
				// updates of the variables to calculate the means Cout and Cin
				updates_for_means_in1(tid); // virtual function for region-based models
				Lout_point = switch_in(Lout_point,tid); // outward local movement
				// switch_in function returns a new Lout_point
				// which is the next point of the former Lout_point
			}
			else
			{
				++Lout_point;
			}
		}
		clean_Lin(tid); // eliminate Lin redundant points
		
		hasInwardEvolutionInThread[tid]=false;
		
		for( list<int>::iterator Lin_point = Splited_Lin[tid]->begin(); !Lin_point.end(); )
		{
			if(compute_external_speed_Fd(*Lin_point,tid) < 0)
			{
				hasInwardEvolutionInThread[tid]=true;
				
				// updates of the variables to calculate the means Cout and Cin
				updates_for_means_out1(tid); // virtual function for region-based models
				Lin_point = switch_out(Lin_point,tid); // inward local movement
				// switch_out function returns a new Lin_point
				// which is the next point of the former Lin_point
			}
			else
			{
				++Lin_point;
			}
		}
		
		clean_Lout(tid); // eliminate Lout redundant points
	}
		
	Lout.collectList(&Splited_Lout,sublistHead, sublistHeadPosition,0,numThreads );
	Lin.collectList(&Splited_Lin,sublistHead, sublistHeadPosition,1,numThreads);
	
	//Recalculate thread private variables
	for(int i=0; i < numThreads; i++)
	{
		if(hasOutwardEvolutionInThread[i])
		{
			hasOutwardEvolution = true;
			break;
		}
	}
	for(int i=0; i < numThreads; i++)
	{
		if(hasInwardEvolutionInThread[i])
		{
			hasInwardEvolution = true;
			break;
		}
	}
	//Collect all counters of each thread
	update_for_means_global();
	
	iteration++;
	return;
}
\end{lstlisting}

\subsubsection{Segundo ciclo}


\begin{lstlisting}
void ActiveContour::do_one_iteration_in_cycle2()
{
	int offset,tid,listSize;
	lists_length = 0;
	
	Lout.splitList(Splited_Lout,numThreads,sublistHead, sublistHeadPosition,0);
	Lin.splitList(Splited_Lin,numThreads,sublistHead, sublistHeadPosition,1);
	
	#pragma omp parallel private(tid,offset) reduction(+:listSize)
	{
		listSize=0;
		tid = omp_get_thread_num();
		
		// scan through Lout with a conditional increment
		for( list<int>::iterator Lout_point = Splited_Lout[tid]->begin(); !Lout_point.end(); )
			{
			offset = *Lout_point;
			
			if( compute_internal_speed_Fint(offset) > 0 )
			{
				// updates of the variables to calculate the means Cout and Cin
				updates_for_means_in2(offset,tid); // virtual function for region-based models
				Lout_point = switch_in(Lout_point,tid); // outward local movement
				// switch_in function returns a new Lout_point
				// which is the next point of the former Lout_point
			}
			else
			{
				listSize++;
				++Lout_point;
			}
		}
		
		clean_Lin(tid); // eliminate Lin redundant points
		
		// scan through Lin with a conditional increment
		for( list<int>::iterator Lin_point = Splited_Lin[tid]->begin(); !Lin_point.end(); )
		{
			offset = *Lin_point;
			
			if( compute_internal_speed_Fint(offset) < 0 )
			{
				// updates of the variables to calculate the means Cout and Cin
				updates_for_means_out2(offset,tid); // virtual function for region-based models
				Lin_point = switch_out(Lin_point,tid); // inward local movement
				// switch_out function returns a new Lin_point
				// which is the next point of the former Lin_point
			}
			else
			{
				listSize++;
				++Lin_point;
			}
		}
		clean_Lout(tid); // eliminate Lout redundant points
	}
	
	Lout.collectList(&Splited_Lout,sublistHead, sublistHeadPosition,0,numThreads );
	Lin.collectList(&Splited_Lin,sublistHead, sublistHeadPosition,1,numThreads);
	
	lists_length = listSize;
	
	//Collect all counters of each thread
	update_for_means_global();
	
	iteration++;
	return;
}
\end{lstlisting}

\end{appendices}
