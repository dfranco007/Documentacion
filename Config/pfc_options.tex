%
% Paquetes que pueden serte de utilidad (rec = recomendado, opc = opcional)
%
\usepackage{fancyhdr}          % (rec)  permite cambiar varios par�metros de las cabeceras y pi�s de p�gina
\usepackage{courier}           % (opc)  usa esta fuente por defecto
\usepackage{setspace}          % (opc)  permite cambiar el espaciado entre l�neas
\usepackage{longtable}         % (opc)  permite que las tablas ocupen varias p�ginas
\usepackage{lscape}            % (opc)  permite el uso del comando \landscape, para poner algo apaisado
\usepackage{color}             % (opc)  varios comandos relativos al color (como \color)
\usepackage{rotating}          % (opc)  permite rotar PSs y EPSs
\usepackage{textcomp}          % (opc)  permite incluir el s�mbolo del euro, con \texteuro
\usepackage{minitoc}           % (opc)  permite incluir ToCs (�ndice de materias) para cada cap�tulo
\usepackage{epsf}              % (opc)  permite ciertas manipulaciones a EPSs
\usepackage[absolute]{textpos} % (rec)  permite posicionado arbitrario de texto (necesario para la portada)
\usepackage{srcltx}            % (opc)  permite pasar del .dvi al .tex
\usepackage[Config/spanish5, es-tabla]{babel}   % (rec)  da soporte para castellano a LaTeX
\usepackage[latin1]{inputenc}  % (opc)  permite introducir caracteres como �, etc, en el input
\usepackage{caption}
\usepackage{subcaption}
\usepackage{float}
\usepackage{amssymb}
\usepackage{booktabs}% http://ctan.org/pkg/booktabs
\usepackage{listings}
\usepackage{color}
\usepackage{slashbox}
\usepackage{graphicx}
\usepackage{multirow}

\definecolor{dkgreen}{rgb}{0,0.6,0}
\definecolor{gray}{rgb}{0.5,0.5,0.5}
\definecolor{mauve}{rgb}{0.58,0,0.82}

\lstset{frame=tb,
	language=C++,
	aboveskip=3mm,
	belowskip=3mm,
	showstringspaces=false,
	columns=flexible,
	basicstyle={\small\ttfamily},
	numbers=none,
	numberstyle=\tiny\color{gray},
	keywordstyle=\color{blue},
	commentstyle=\color{dkgreen},
	stringstyle=\color{mauve},
	breaklines=true,
	breakatwhitespace=true,
	tabsize=3
}

\newcommand{\tabitem}{~~\llap{\textbullet}~~}
%
% Settings para los m�rgenes. Descomenta y modifica si sabes lo que haces. N�tese
% que a los valores dados se les a�ade una pulgada extra. Los valores dados son los
% predeterminados para papel A4 y el estilo itsas_pfc.cls.
%
%\setlength{\oddsidemargin}{10pt}     % m�rgen izquierdo para p�ginas impares (izquierda)
%\setlength{\evensidemargin}{52pt}    % m�rgen izquierdo para p�ginas pares (derecha)
%\setlength{\textwidth}{390pt}        % anchura del cuerpo de texto

%
% Recomendado para mejorar la colocaci�n autom�tica de las figuras.
% (tomado de http://dcwww.camp.dtu.dk/~schiotz/comp/LatexTips/LatexTips.html#captfont)
%
\renewcommand{\topfraction}{0.85}
\renewcommand{\textfraction}{0.1}
\renewcommand{\floatpagefraction}{0.75}
%
% Espacio entre el borde superior de la p�gina y donde comienza el texto (ah� van las
% cabeceras). LaTeX se queja si usamos el paquete fanchyhdr y headheight es menor de 15pt
%
\headheight 15pt

%
% Para el paquete textpos (usado para la portada)
%
\setlength{\TPHorizModule}{\paperwidth}
\setlength{\TPVertModule}{\paperheight}
\newcommand{\tb}[4]{\begin{textblock}{#1}[0.5,0.5](#2,#3)\begin{center}#4\end{center}\end{textblock}}

%
% Aqu� puedes definir tus comandos.
% 
% \newcommand{cmd}[args]{def}
%
% cmd  = el comando a definir (p.e. \cadena)
% args = el n�mero de argumentos
% def  = la definici�n, sustituyendo #1, #2... por el primer, segundo... argumento
%
% Por ejemplo:
%
% \newcommand{\agua}[1]{H\ensuremath{_#1}O}
%
% Cada vez que escribamos "\agua{33}", en el output saldr�: "H33O" (con el 33 como sub�ndice)
%

%\newcommand{\algo}{algo}

%
% Aqu� puedes instruir a LaTeX de por d�nde cortar las palabras que �l autom�ticamente
% no sepa. P.e., para cortar "gnomonly" solo por donde se se�ala con guiones (-).
%
\hyphenation{gno-mon-ly} 
 
%
% Que las primeras p�ginas sean numeradas con n�meros romanos.
% M�s adelante se cambiar� de nuevo a ar�bicos.
%
\pagenumbering{Roman}
